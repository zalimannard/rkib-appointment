\documentclass[a4paper,article]{article}

% Шрифты
\usepackage{fontspec}
\usepackage[14pt]{extsizes}
\setmainfont{Times New Roman}

% Языки
% Русский обязательно идёт вторым. Иначе не работают переносы
\usepackage[english, russian]{babel}

% Параметры страницы
\usepackage[left=3cm, top=2cm, right=1.5cm, bottom=2cm]{geometry}
\usepackage[onehalfspacing]{setspace}

% Параметры текста
% По умолчанию LaTeX не делает отступ после \section. Вроде как оно и не надо, но в тексте ВКР пусть лучше будет. В требованиях отступ описан. Этот пакет своим наличием добавляет этот отступ
\usepackage{indentfirst}
% По умолчанию абзацный отстум меньше требуемого. Задаём конкретный
\setlength{\parindent}{1.25cm}

% Ссылки
\usepackage{color}
\definecolor{Black}{RGB}{0,0,0}
% Без colorlinks вокруг ссылок появляются рамки, недопустимые в ВКР
% Если не зачернить ссылки, то в оглавлении и в других местах будут яркие цвета
\usepackage[colorlinks, allcolors=Black]{hyperref}
% Шрифт для URL
\urlstyle{rm}

% Пункты оглавления
\usepackage{titlesec}
\titleformat{\section}
{\centering\normalfont\bfseries}{\thesection. }{0em}{}
\titleformat{\subsection}
{\centering\normalfont\bfseries}{\thesubsection. }{0em}{}
% Список использованных источников
\addto\captionsrussian{\def\refname{Список использованных источников}}

% Таблицы
\usepackage{multicol}
\usepackage{xltabular}

% Перечисления
\usepackage{enumitem}

\begin{document}
    \begin{titlepage}
        \begin{center}
            {\bfseries Министерство науки и высшего образования Российской Федерации \\
            Федеральное государственное автономное образовательное учреждение \\
            высшего образования \\
            <<КАЗАНСКИЙ (ПРИВОЛЖСКИЙ) ФЕДЕРАЛЬНЫЙ УНИВЕРСИТЕТ>>}
        \end{center}

        \begin{center}
            ИНСТИТУТ ВЫЧИСЛИТЕЛЬНОЙ МАТЕМАТИКИ И ИНФОРМАЦИОННЫХ ТЕХНОЛОГИЙ
        \end{center}

        \begin{center}
            КАФЕДРА АНАЛИЗА ДАННЫХ И ТЕХНОЛОГИЙ ПРОГРАММИРОВАНИЯ
        \end{center}

        \begin{center}
            Направление: 09.03.03 – Прикладная информатика
        \end{center}

        \vspace{0mm}

        \begin{center}
            ВЫПУСКНАЯ КВАЛИФИКАЦИОННАЯ РАБОТА \\
            {\bfseries СИСТЕМА ЗАПИСИ НА ПРИЁМ В МЕДИЦИНСКОЕ УЧРЕЖДЕНИЕ}
        \end{center}

        \vfill

        \begin{xltabular}{\textwidth} {
                >{\hsize=0.5\hsize} X
                >{\hsize=0.5\hsize} X }
            \bfseries{Работа завершена:} & \\
            Студент 4 курса & \\
            группы 09-951 & \\
            <<\underline{\hspace{1cm}}>>\underline{\hspace{3cm}} 2023г. & \underline{\hspace{3cm}}/Колесников Д.А. \\
            & \\
            \bfseries{Работа допущена к защите:} & \\
            Научный руководитель & \\
            старший преподаватель & \\
            <<\underline{\hspace{1cm}}>>\underline{\hspace{3cm}} 2023г. & \underline{\hspace{3cm}}/Еникеев И.А. \\
            & \\
            \multicolumn{2}{l}{Заведующий кафедрой анализа данных} \\
            и технологий программирования & \\
            <<\underline{\hspace{1cm}}>>\underline{\hspace{3cm}} 2023г. & \underline{\hspace{3cm}}/Бандеров В.В. \\
        \end{xltabular}

        \vspace{0mm}

        \begin{center}
            Казань — 2023
        \end{center}
    \end{titlepage}

    \newpage

    \setcounter{page}{2}

    \tableofcontents

    \newpage

    \section*{Глоссарий}
    \addcontentsline{toc}{section}{Глоссарий}

    \textbf{РКИБ} - ГАУЗ <<Республиканская клиническая инфекционная больница имени Агафонова>>

    \newpage

    \section*{Введение}
    \addcontentsline{toc}{section}{Введение}

    Медицинской отрасли не хватает информатизации. Это устоявшаяся сфера, однако она не соответствует современности: никто не хочет стоять в очередях, вручную заполнять документы, потерять выданный рецепт. В 2018 году Правительство РФ создало национальный проект <<Здравоохранение>>, направленный на улучшение медицины. Одна из задач проекта - перенос Минздравом в электронный формат части услуг: выдача рецептов, запись к врачу, подача заявления на полис, хранение медицинских документов \cite{natsproektzdravoohranenie}.

	На деле перевели услуги в электронный формат не везде. В России 30~000 медицинских учреждений и нужно много времени для внедрения системы в каждое из них. При этом нужно учитывать особенности каждого учреждения - то, что подойдёт больнице в Москве, может не подойти поликлинике из маленького города.

    Одно из учреждений так и не перешедших в электронный формат - Республиканская клиническая инфекционная больница имени Агафонова (РКИБ). Сейчас в неё активно внедряюся цифровые технологии, например - оплата услуг. Но некоторые элементы остаются прежними, в частности, запись на приём. С ней то нам и предстоит разобраться.

    \textbf{Цель:} Информатизировать запись на приём в РКИБ

    \textbf{Задачи:}

    \begin{enumerate}[nolistsep, left=0.7cm, ]
        \item Изучить предметную область
        \item Проанализировать существующие аналоги
        \item Создать техническое задание
        \item Спроектировать части будущей системы
        \item Реализовать спроектированные части системы
        \item Внедрить разработанную систему в работу РКИБ
    \end{enumerate}

    \textbf{Объект исследования:} Информатизация в системе здравоохранения

    \textbf{Предмет исследования:} Запись на приём в медицинское учреждение

    \textbf{Структура:} В первой части работы анализируется предметная область и определяется необходимый функционал. Во второй главе будущая система проектируется, в третьей - реализуется. Заключительная часть - введение готовой системы в эксплуатацию.

    \newpage

    \section{Предметная область}

    \subsection{Основные сведения}

    \subsection{Существующие решения}

    \subsection{Техническое задание}

    \pagestyle{plain}

    \newpage

    \section{Проектирование}

    \subsection{База данных}

    \subsection{Серверная часть}

    \subsection{Клиентская часть}

    \newpage

    \section{Реализация}

    \subsection{Выбор технологий}

    \subsection{База данных}

    \subsection{Серверная часть}

    \subsection{Клиентская часть}

    \newpage

    \section{Выпуск}

    \newpage

    \section*{Заключение}
    \addcontentsline{toc}{section}{Заключение}

    \newpage

    \addcontentsline{toc}{section}{Список использованных источников}

    \begin{thebibliography}{}
        \bibitem{natsproektzdravoohranenie} Федеральный проект <<Создание единого цифрового контура в здравоохранении на основе единой государственной информационной системы в сфере здравоохранения (ЕГИСЗ)>> - 2019 - 9 августа [Электронный ресурс] - URL: \url{https://minzdrav.gov.ru/poleznye-resursy/natsproektzdravoohranenie/tsifra/} (Дата обращения: 13.12.2022)
    \end{thebibliography}

\end{document}
