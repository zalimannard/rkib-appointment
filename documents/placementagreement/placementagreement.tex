\documentclass[a4paper,article]{article}

% Шрифты
\usepackage{fontspec}
\usepackage[14pt]{extsizes}
\setmainfont{Times New Roman}

% Языки
% Русский обязательно идёт вторым. Иначе не работают переносы
\usepackage[russian, english]{babel}

% Параметры страницы
\usepackage[left=3cm, top=2cm, right=1.5cm, bottom=2cm]{geometry}

% Параметры текста
% По умолчанию LaTeX не делает отступ после \section. Вроде как оно и не надо, но в тексте ВКР пусть лучше будет. В требованиях отступ описан. Этот пакет своим наличием добавляет этот отступ
\usepackage{indentfirst}
% По умолчанию абзацный отстум меньше требуемого. Задаём конкретный
\setlength{\parindent}{1.25cm}

% Ссылки
\usepackage{color}
\definecolor{Black}{RGB}{0,0,0}
% Без colorlinks вокруг ссылок появляются рамки, недопустимые в ВКР
% Если не зачернить ссылки, то в оглавлении и в других местах будут яркие цвета
\usepackage[colorlinks, allcolors=Black]{hyperref}
% Шрифт для URL
\urlstyle{rm}

% Пункты оглавления
\usepackage{titlesec}
\titleformat{\section}
{\centering\normalfont\bfseries}{\thesection. }{0em}{}
\titleformat{\subsection}
{\centering\normalfont\bfseries}{\thesubsection. }{0em}{}
\titleformat{\subsubsection}
{\centering\normalfont\bfseries}{\thesubsubsection. }{0em}{}
\titleformat{\paragraph}
{\normalfont\normalsize\bfseries\centering}{\theparagraph. }{0em}{}
% Список использованных источников
\addto\captionsrussian{\def\refname{Список использованных источников}}

% Нумерация
\setcounter{secnumdepth}{4}

% Таблицы
\usepackage{multicol}
\usepackage{xltabular}

% Перечисления
\usepackage{enumitem}
% Выравнивание списков
\setlist[itemize]{leftmargin=1.25cm}
\setlist[enumerate]{leftmargin=1.25cm}

% Картинки
% Вставка картинок правильная
\usepackage{graphicx}
% "Плавающие" картинки
\usepackage{float}
% Обтекание фигур (таблиц, картинок и прочего)
\usepackage{wrapfig}
% Папка для всех картинок файла
\graphicspath{{images/}}
% Точка в конце названий объекта вместо двоеточия. Например, для Рисунков
\usepackage[labelsep=period]{caption}
% Заменяет 'Рис. 1' на 'Рисунок 1'
\makeatletter
\renewcommand{\fnum@figure}{Рисунок \thefigure}
\makeatother
% Рамка
\usepackage{adjustbox}

% Листинги
\usepackage{listings}
\usepackage{caption}
\usepackage{xcolor}
\definecolor{intellijgreen}{RGB}{106,135,89}
\definecolor{intellijorange}{RGB}{204,120,50}
\lstset{
    frame=single,
    breaklines=true,
    captionpos=t,
    numbersep=-14pt,
    numbers=left,
    numberstyle=\fontsize{12}{14}\selectfont,
    stepnumber=1,
    language=Java,
    basicstyle=\fontsize{12}{14}\selectfont\ttfamily,
    keywordstyle=\color{intellijorange},
    commentstyle=\color{gray},
    stringstyle=\color{intellijgreen},
    showstringspaces=false,
}

% Переносы
\usepackage{microtype}

% Приложения
\usepackage{appendix}

\pagestyle{empty}

\begin{document}
\begin{sloppypar}
    \begin{center}
        \textbf{РАЗРЕШЕНИЕ \\
        на размещение выпускной квалификационной работы \\
        в электронно-библиотечной системе КФУ} \\
    \end{center}
    
    \vspace{-1em}
    
    \noindent
    1. Я, \hfill \underline{\hspace{15.9cm}} являющийся обучающимся \hfill \underline{\hspace{11cm}} \noindent\underline{\hspace{\textwidth}} федерального государственного автономного образовательного учреждения высшего образования «Казанский (Приволжский) федеральный университет» (далее – КФУ), разрешаю КФУ безвозмездно воспроизводить и размещать выпускную квалификационную работу в сети Интернет в электронно – библиотечной системе КФУ, на официальном портале КФУ на тему:
    
    \noindent
    <<\underline{\hspace{16cm}}
    \underline{\hspace{16cm}}>>
    
    
    \begin{itemize}[nolistsep]
        \item в полном объёме;
        \item с изъятием, содержащих производственные, технические, экономические, организационные и другие сведения, в том числе о результатах интеллектуальной деятельности в научно-технической сфере, которые имеют действительную или потенциальную коммерческую ценность в силу неизвестности их третьим лицам. \newline
    \end{itemize}
    
    \noindent
    2. В случае непредставления мною в установленные сроки электронной копии ВКР с изъятием, содержащих производственные, технические, экономические, организационные и другие сведения, в том числе о результатах интеллектуальной деятельности в научно-технической сфере, которые имеют действительную или потенциальную коммерческую ценность в силу неизвестности их третьим лицам в формате PDF уведомлен, что в электронно-библиотечной системе будет размещена полная версия ВКР. \newline
    
    \noindent
    3. Я подтверждаю, что выпускная квалификационная работа написана мною лично, в соответствии с правилами академической этики и не нарушает авторских прав иных лиц. \newline
    
    \noindent
    4. Я разрешаю размещение выпускной квалификационной работы в электронно-библиотечной системе КФУ с момента подписания мною настоящего разрешения.
    
    \vfill
    
    \noindent
    <<\underline{\hspace{1cm}}>>\underline{\hspace{3cm}} 2023 г. \hfill \underline{\hspace{3cm}}/\underline{\hspace{3cm}}
\end{sloppypar}
\end{document}
