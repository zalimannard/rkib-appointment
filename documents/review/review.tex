\documentclass[a4paper,article]{article}

% Шрифты
\usepackage{fontspec}
\usepackage[12pt]{extsizes}
\setmainfont{Times New Roman}

% Языки
% Русский обязательно идёт вторым. Иначе не работают переносы
\usepackage[english, russian]{babel}

% Параметры страницы
\usepackage[left=3cm, top=2cm, right=1.5cm, bottom=2cm]{geometry}
\usepackage[onehalfspacing]{setspace}

% Параметры текста
% По умолчанию LaTeX не делает отступ после \section. Вроде как оно и не надо, но в тексте ВКР пусть лучше будет. В требованиях отступ описан. Этот пакет своим наличием добавляет этот отступ
\usepackage{indentfirst}
% По умолчанию абзацный отстум меньше требуемого. Задаём конкретный
\setlength{\parindent}{1.25cm}

% Ссылки
\usepackage{color}
\definecolor{Black}{RGB}{0,0,0}
% Без colorlinks вокруг ссылок появляются рамки, недопустимые в ВКР
% Если не зачернить ссылки, то в оглавлении и в других местах будут яркие цвета
\usepackage[colorlinks, allcolors=Black]{hyperref}
% Шрифт для URL
\urlstyle{rm}

% Пункты оглавления
\usepackage{titlesec}
\titleformat{\section}
{\centering\normalfont\bfseries}{\thesection. }{0em}{}
\titleformat{\subsection}
{\centering\normalfont\bfseries}{\thesubsection. }{0em}{}
\titleformat{\subsubsection}
{\centering\normalfont\bfseries}{\thesubsubsection. }{0em}{}
\titleformat{\paragraph}
{\normalfont\normalsize\bfseries\centering}{\theparagraph. }{0em}{}
% Список использованных источников
\addto\captionsrussian{\def\refname{Список использованных источников}}

% Нумерация
\setcounter{secnumdepth}{4}

% Таблицы
\usepackage{multicol}
\usepackage{xltabular}

% Перечисления
\usepackage{enumitem}
% Выравнивание списков
\setlist[itemize]{leftmargin=1.25cm}
\setlist[enumerate]{leftmargin=1.25cm}

% Картинки
% Вставка картинок правильная
\usepackage{graphicx}
% "Плавающие" картинки
\usepackage{float}
% Обтекание фигур (таблиц, картинок и прочего)
\usepackage{wrapfig}
% Папка для всех картинок файла
\graphicspath{{images/}}
% Точка в конце названий объекта вместо двоеточия. Например, для Рисунков
\usepackage[labelsep=period]{caption}
% Заменяет 'Рис. 1' на 'Рисунок 1'
\makeatletter
\renewcommand{\fnum@figure}{Рисунок \thefigure}
\makeatother
% Рамка
\usepackage{adjustbox}

% Листинги
\usepackage{listings}
\usepackage{caption}
\usepackage{xcolor}
\definecolor{intellijgreen}{RGB}{106,135,89}
\definecolor{intellijorange}{RGB}{204,120,50}
\lstset{
    frame=single,
    breaklines=true,
    captionpos=t,
    numbersep=-14pt,
    numbers=left,
    numberstyle=\fontsize{12}{14}\selectfont,
    stepnumber=1,
    language=Java,
    basicstyle=\fontsize{12}{14}\selectfont\ttfamily,
    keywordstyle=\color{intellijorange},
    commentstyle=\color{gray},
    stringstyle=\color{intellijgreen},
    showstringspaces=false,
}

% Переносы
\usepackage{microtype}

% Приложения
\usepackage{appendix}

\pagestyle{empty}

\begin{document}
\begin{sloppypar}
    \begin{center}
        \textbf{Министерство науки и высшего образования Российской Федерации \\
        Федеральное государственное автономное образовательное учреждение \\
        высшего образования\\
        <<КАЗАНСКИЙ (ПРИВОЛЖСКИЙ) ФЕДЕРАЛЬНЫЙ УНИВЕРСИТЕТ>>} \\
    \end{center}
    
    \begin{center}
        ИНСТИТУТ ВЫЧИСЛИТЕЛЬНОЙ МАТЕМАТИКИ И \\
        ИНФОРМАЦИОННЫХ ТЕХНОЛОГИЙ
    \end{center}
    
    \begin{center}
        КАФЕДРА АНАЛИЗА ДАННЫХ И \\
        ТЕХНОЛОГИЙ ПРОГРАММИРОВАНИЯ
    \end{center}

    \begin{center}
        \textbf{РЕЦЕНЗИЯ \\
        на выпускную квалификационную работу \\
        обучающегося 09-951 группы 4 курса очной форме обучения по направлению \\
        09.03.03 Прикладная информатика \\
        Колесникова Дмитрия Александровича} \\
    \end{center}

    Тема ВКР: Система записи на приём в медицинское учреждение
    
    Рецензент: Старший преподаватель кафедры программной инженерии института информационных технологий и интеллектуальных систем Якупов Азат Шавкатович \newline
    
    Выпускная квалификационная работа Колесникова Д.А. фокусируется на создании системы, цель которой - оптимизация процессов в медицинских организациях. В частности, система будет облегчать составление расписания для врачей и процесс записи пациентов. Данная система способствует экономии времени на синхронизацию и поиск информации.
    
    Исследование включает в себя обзор предметной области, основываясь на требованиях республиканской клинической инфекционной больницы имени профессора А.Ф. Агафонова. Разработанное приложение состоит из трех компонентов: базы данных, серверной и клиентской частей. База данных разработана в соответствии с поставленными задачами, используя Oracle в качестве СУБД. Серверная часть - это REST API, реализованный на основе фреймворка Spring. Клиентская часть – веб-интерфейс, созданный с помощью фреймворка Vue.js и языка TypeScript. Такой выбор технологий был сделан на основе требований проекта и оптимально подходит для решения поставленных задач.
    
    Важно отметить тщательно продуманный подход к тестированию кода, реализованный с помощью методики TDD и технологий JUnit5, AssertJ, RestAssured. Все результаты тестирования были визуализированы с использованием Allure. Всё это свидетельствует о стремлении Колесникова Д.А. к созданию качественного продукта.
    
    Работа Колесникова Д.А. демонстрирует глубокие знания и понимание выбранных технологий, способность к самостоятельному изучению новых инструментов и методик работы. В процессе выполнения работы он проявил себя как самостоятельного, ответственного исследователя, заинтересованного в обучении и развитии.
    
    По итогам анализа представленной работы, учитывая всестороннюю разработку темы, планомерный подход к выполнению проекта, качество кода, ясное и последовательное представление материала, предлагаю оценить выпускную квалификационную работу Колесникова Д.А. на <<отлично>>. \newline
    
    \textbf{Оценивание параметров текста ВКР}

    \begin{xltabular}{\textwidth} { |
            >{\hsize=0.70\hsize} X |
            >{\hsize=0.30\hsize} X | }

        \hline
        \textbf{Параметр} 
        & \textbf{Оценка} \\
        \hline
        Актуальность темы работы
        & Отлично \\
        \hline
        Раскрытие темы ВКР
        & Отлично \\
        \hline
        Уровень теоретической проработки проблемы
        & Отлично \\
        \hline
        Качество анализа проблемы, достоверность выводов и обоснованность выдвигаемых проектных решений
        & Отлично \\
        \hline
        Самостоятельность и творческий подход к разработке темы
        & Отлично \\
        \hline
        Грамотность написания и оформления работы, его соответствие установленным стандартам
        & Отлично \\
        \hline
        Степень выполнения задач и реализация цели ВКР
        & Отлично \\
        \hline
        Соблюдение графика работы
        & Отлично \\
        \hline
    \end{xltabular}

    \begin{xltabular}{\textwidth} {
            >{\hsize=0.60\hsize} X
            >{\hsize=0.40\hsize} X }
        Старший преподаватель кафедры программной & \\
        инженерии института информационных технологий  & \\
        и интеллектуальных систем  & \underline{\hspace{3cm}}/Якупов А.Ш. \\
        & \\
        & \\
        & \hfil Дата \underline{\hspace{3cm}} \\
    \end{xltabular}
\end{sloppypar}
\end{document}
