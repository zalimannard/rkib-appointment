\documentclass[a4paper,article]{article}

% Шрифты
\usepackage{fontspec}
\usepackage[14pt]{extsizes}
\setmainfont{Times New Roman}

% Языки
% Русский обязательно идёт вторым. Иначе не работают переносы
\usepackage[english, russian]{babel}

% Параметры страницы
\usepackage[left=1.5cm, top=1.5cm, right=1.5cm, bottom=1.5cm]{geometry}
\usepackage[onehalfspacing]{setspace}

% Параметры текста
% По умолчанию LaTeX не делает отступ после \section. Вроде как оно и не надо, но в тексте ВКР пусть лучше будет. В требованиях отступ описан. Этот пакет своим наличием добавляет этот отступ
\usepackage{indentfirst}
% По умолчанию абзацный отстум меньше требуемого. Задаём конкретный
\setlength{\parindent}{1.25cm}

% Ссылки
\usepackage{color}
\definecolor{Black}{RGB}{0,0,0}
% Без colorlinks вокруг ссылок появляются рамки, недопустимые в ВКР
% Если не зачернить ссылки, то в оглавлении и в других местах будут яркие цвета
\usepackage[colorlinks, allcolors=Black]{hyperref}
% Шрифт для URL
\urlstyle{rm}

% Пункты оглавления
\usepackage{titlesec}
\titleformat{\section}
{\centering\normalfont\bfseries}{\thesection. }{0em}{}
\titleformat{\subsection}
{\centering\normalfont\bfseries}{\thesubsection. }{0em}{}
\titleformat{\subsubsection}
{\centering\normalfont\bfseries}{\thesubsubsection. }{0em}{}
\titleformat{\paragraph}
{\normalfont\normalsize\bfseries\centering}{\theparagraph. }{0em}{}
% Список использованных источников
\addto\captionsrussian{\def\refname{Список использованных источников}}

% Нумерация
\setcounter{secnumdepth}{4}

% Таблицы
\usepackage{multicol}
\usepackage{xltabular}

% Перечисления
\usepackage{enumitem}
% Выравнивание списков
\setlist[itemize]{leftmargin=1.25cm}
\setlist[enumerate]{leftmargin=1.25cm}

% Картинки
% Вставка картинок правильная
\usepackage{graphicx}
% "Плавающие" картинки
\usepackage{float}
% Обтекание фигур (таблиц, картинок и прочего)
\usepackage{wrapfig}
% Папка для всех картинок файла
\graphicspath{{images/}}
% Точка в конце названий объекта вместо двоеточия. Например, для Рисунков
\usepackage[labelsep=period]{caption}
% Заменяет 'Рис. 1' на 'Рисунок 1'
\makeatletter
\renewcommand{\fnum@figure}{Рисунок \thefigure}
\makeatother
% Рамка
\usepackage{adjustbox}

% Листинги
\usepackage{listings}
\usepackage{caption}
\usepackage{xcolor}
\definecolor{intellijgreen}{RGB}{106,135,89}
\definecolor{intellijorange}{RGB}{204,120,50}
\lstset{
    frame=single,
    breaklines=true,
    captionpos=t,
    numbersep=-14pt,
    numbers=left,
    numberstyle=\fontsize{12}{14}\selectfont,
    stepnumber=1,
    language=Java,
    basicstyle=\fontsize{12}{14}\selectfont\ttfamily,
    keywordstyle=\color{intellijorange},
    commentstyle=\color{gray},
    stringstyle=\color{intellijgreen},
    showstringspaces=false,
}

% Переносы
\usepackage{microtype}

% Приложения
\usepackage{appendix}

\pagestyle{empty}

\begin{document}
    \begin{sloppypar}
        \begin{center}
            Таблица компетенций
        \end{center}
        
        
        \begin{xltabular}{\textwidth} { |
                >{\hsize=0.10\hsize} X |
                >{\hsize=0.55\hsize} X |
                >{\hsize=0.35\hsize} X | }
            
            \hline
            УК-1
            & Способен осуществлять поиск, критический анализ и синтез информации, применять системный подход для решения поставленных задач
            & Был проведён анализ предметной области и существующих решений. Страницы 8-16 \\
            \hline
            УК-2
            & Способен определять круг задач в рамках поставленной цели и выбирать оптимальные способы их решения, исходя из действующих правовых норм, имеющихся ресурсов и ограничений
            & На основе цели были сформулированы задачи. Страница 8 \\
            \hline
            УК-3
            & Способен осуществлять социальное взаимодействие и реализовывать свою роль в команде
            & Предметная область проанализирована с участием сотрудников настоящей больницы. Страницы 8-16 \\
            \hline
            УК-4
            & Способен осуществлять деловую коммуникацию в устной и письменной формах на государственном языке Российской Федерации и иностранном(ых) языке(ах)
            & ВКР написана на русском языке, аннотация продублирована на английском. Страницы 4-5 \\
            \hline
            УК-5
            & Способен воспринимать межкультурное разнообразие общества в социально-историческом, этическом и философском контекстах
            & Налажена коммуникация с потенциальными пользователями \\
            \hline
            УК-6
            & Способен управлять своим временем, выстраивать и реализовывать траекторию саморазвития на основе принципов образования в течение всей жизни 
            & Выстраивает план разработки и выполняет его в срок \\
            \hline
            УК-7
            & Способен поддерживать должный уровень физической подготовленности для обеспечения полноценной социальной и профессиональной деятельности
            & Во время создания проекта было определено разграничение умственной и физической активности \\
            \hline
            УК-8
            & Способен создавать и поддерживать безопасные условия жизнедеятельности, в том числе при возникновении чрезвычайных ситуаций
            & Использовал интернет-хранилище для контроля версий, что не даст пропасть разработкам \\
            \hline
            ОПК-1
            & Способен применять естественнонаучные и общеинженерные знания, методы математического анализа и моделирования, теоретического и экспериментального исследования в профессиональной деятельности
            & Компетенция освоена во время реализации каждого модуля \\
            \hline
            ОПК-2
            & Способен использовать современные информационные технологии и программные средства, в том числе отечественного производства, при решении задач профессиональной деятельности
            & Система разработана с использованием большого количества разнообразных технологий \\
            \hline
            ОПК-3
            & Способен решать стандартные задачи профессиональной деятельности на основе информационной и библиографической культуры с применением информационно-коммуникационных технологий и с учетом основных требований информационной безопасности
            & Изучена информация по лучшим практикам используемых технологий \\
            \hline
            ОПК-4
            & Способен участвовать в разработке стандартов, норм и правил, а также технической документации, связанной с профессиональной деятельностью
            & Были описаны функциональные и нефункциональные требования, создана документация по API. Страницы 12-13, 22-23 \\
            \hline
            ОПК-5
            & Способен инсталлировать программное и аппаратное обеспечение для информационных и автоматизированных систем
            & Разработанное приложение разворачивается с использованием Docker. Страница 58, Приложение 2 \\
            \hline
            ОПК-6
            & Способен анализировать и разрабатывать организационно-технические и экономические процессы с применением методов системного анализа и математического моделирования
            & Во время разработки учитывались требования разработки \\
            \hline
            ОПК-7
            & Способен разрабатывать алгоритмы и программы, пригодные для практического применения
            & Необходимый спроектированный функционал был реализован. Страницы 31-58 \\
            \hline
            ОПК-8
            & Способен принимать участие в управлении проектами создания информационных систем на стадиях жизненного цикла 
            & Участвовал в каждом этапе жизненного цикла разрабатываемой системы \\
            \hline
            ОПК-9
            & Способен принимать участие в реализации профессиональных коммуникаций с заинтересованными участниками проектной деятельности и в рамках проектных групп
            & Во время разработки были коммуникации с научным руководителем и потенциальными пользователями \\
            \hline
            ПК-1
            & Проверка работоспособности и рефракторинг кода программного обеспечения, интеграция программных модулей и компонент и верификация выпусков программного обеспечения
            & Проведено тестирование, переработка существующего кода для добавления новых функций \\
            \hline
            ПК-2
            & Мониторинг функционирования интеграционного решения в соответствии с трудовым заданием, работа с обращениями пользователей по вопросам функционирования интеграционного решения в соответствии с трудовым заданием
            & Организована связь с потенциальными пользователями системы \\
            \hline
            ПК-3
            & Проверка и отладка программного кода, тестирование информационных ресурсов с точки зрения логической целостности (корректность ссылок, работа элементов форм)
            & Проведено ручное тестирование, автоматические интеграционное и UI тестирования. Страницы 47-49, 57 \\
            \hline
            ПК-4
            & Ведение информационных баз данных 
            & В системе используется база данных Oracle. Страницы 35-36 \\
            \hline
            ПК-5
            & Обеспечения функционирования баз данных
            & Написан код для взаимодействия серверной части и базы данных. Страницы 35-37 \\
            \hline
            ПК-6
            & Педагогическая деятельность по проектированию и реализации общеобразовательных программ
            & Помощь одногруппникам в разработке их проектов \\
            \hline
            ПК-7
            & Способность находить организационно-управленческие решения в нестандартных ситуациях и готовность нести за них ответственность
            & Решены все проблемы, появляющиеся в процессе разработки системы \\
            \hline
            ПК-8
            & Способность к коммуникации, восприятию информации, умение логически верно, аргументировано и ясно строить устную и письменную речь на русском языке для решения задач профессиональной коммуникации
            & ВКР написана на русском языке. Взаимодействие с другими людьми, связанными с разрабатываемой системой, велось на русском языке \\
            \hline
            ПК-9
            & Знанием своих прав и обязанностей как гражданина своей страны, способностью использовать действующее законодательство и другие правовые документы в своей деятельности, демонстрировать готовность и стремление к совершенствованию и развитию общества на принципах гуманизма, свободы и демократии
            & Разработка приложения велась в соответствии с действующим законодательством Российской Федерации \\
            \hline
        \end{xltabular}
    \end{sloppypar}
\end{document}
